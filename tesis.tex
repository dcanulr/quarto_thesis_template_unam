%%%%%%%%%%%%%%%%%%%%%%%%%%%%%%%%%%%%%%%%%%%%%%%%%%%%%%%%%%%%%%%%%%%%%%%%%%%%%%%%
%                         FORMATO DE TESIS UMSNH                               %
%%%%%%%%%%%%%%%%%%%%%%%%%%%%%%%%%%%%%%%%%%%%%%%%%%%%%%%%%%%%%%%%%%%%%%%%%%%%%%%%
% based on Harish Bhanderi's PhD/MPhil template, then Uni Cambridge
% http://www-h.eng.cam.ac.uk/help/tpl/textprocessing/ThesisStyle/
% corrected and extended in 2007 by Jakob Suckale, then MPI-iCBG PhD programme
% and made available through OpenWetWare.org - the free biology wiki

%                     Under GNU License v3
% Adaptado para UNAM: @Tepexic
% ADAPTADO PARA UMSNH:  @arturolp

\documentclass[twoside,11pt]{Latex/Classes/PhDthesisPSnPDF} % "thesisUMSNH" para formato de la UMSNH
%         PUEDEN INCLUIR EN ESTE ESPACIO LOS PAQUETES EXTRA, O BIEN, EN EL ARCHIVO "PhDthesisPSnPDF.cls" EN "./Latex/Classes/"

% Estos paquetes son opcionales y a necesidad de cada quien:
\usepackage{blindtext}                             % Para insertar texto dummy, de ejemplo, pues.
\usepackage{amssymb, amsmath, amsbsy, amsfonts}    % ECUACIONES Y SÍMBOLOS MATEMÁTICOS
\usepackage{listings}                    % PERMITE AGREGAR CÓDIGO DE LENGUAJES  DE PROGRAMACIÓN (DOCUMENTACIÓN EN GOOGLE)
\usepackage{mathdots}                    % para el comando \iddots
\usepackage{mathrsfs}                    % para formato de letra en ecuaciones
\usepackage[round, sort, numbers]{natbib}  % Personalizar la bibliografía a gusto de cada quien

% Note:
% The \blindtext or \Blindtext commands throughout this template generate dummy text
% to fill the template out. These commands should all be removed when 
% writing thesis content.


% This file contains macros that can be called up from connected TeX files
% It helps to summarise repeated code, e.g. figure insertion (see below).

%%%%%%%%%%%%%%%%%%%%%%%%%%%%%%%%%%%%%%%%%%%%%%
%            Colores de la UNAM              %
%%%%%%%%%%%%%%%%%%%%%%%%%%%%%%%%%%%%%%%%%%%%%%
% Para UNAN: Azul Pantone 655C-->(0,3,84) RGB
% Para UMSNH: PANTONE Blue 072 C
\definecolor{Azul}{RGB}{0,3,84}

% Para UNAM: Oro PANTONE 871C-->(137,118,75) RGB
% Para UMNSH: PANTONE 110 C
\definecolor{Oro}{RGB}{137,118,75}


%%%%%%%%%%%%%%%%%%%%%%%%%%%%%%%%%%%%%%%%%%%%%%
%            Comandos para líneas            %
%%%%%%%%%%%%%%%%%%%%%%%%%%%%%%%%%%%%%%%%%%%%%%
%Se define un comando \colorvrule para hacer líneas verticales de color con 3 argumentos: color, ancho, alto
\newcommand{\colorvrule}[3]{
\begingroup\color{#1}\vrule width#2 height#3
\endgroup}

%Se define un comando \colorhrule para hacer líneas horizontales de color con 2 argumentos: color, ancho
\newcommand{\colorhrule}[2]{
\begingroup\color{#1}\hrule height#2
\endgroup}

%%%%%%%%%%%%%%%%%%%%%%%%%%%%%%%%%%%%%%%%%%%%%%
%          Comando para derivadas            %
%%%%%%%%%%%%%%%%%%%%%%%%%%%%%%%%%%%%%%%%%%%%%%
\newcommand{\derivada}[3][]{\ensuremath{\dfrac{\mbox{d}^{#1}#2}{\mbox{d}#3^{#1}}}} 
%primer argumento(opcional): orden de la derivada
%segundo argumento: función a derivar
%tercer argumento: variable respecto a la que se deriva


%%%%%%%%%%%%%%%%%%%%%%%%%%%%%%%%%%%%%%%%%%%%%%
%       Comando para la exponencial          %
%%%%%%%%%%%%%%%%%%%%%%%%%%%%%%%%%%%%%%%%%%%%%%
\newcommand{\e}[1][]{\ensuremath{\mbox{e}^{#1}}}
%primer argumento(opcional): exponente de la exponencial




% insert a centered figure with caption and description
% parameters 1:filename, 2:title, 3:description and label
\newcommand{\figuremacro}[3]{
	\begin{figure}[htbp]
		\centering
		\includegraphics[width=1\textwidth]{#1}
		\caption[#2]{\textbf{#2} - #3}
		\label{condicion}
	\end{figure}
}

% insert a centered figure with caption and description AND WIDTH
% parameters 1:filename, 2:title, 3:description and label, 4: textwidth
% textwidth 1 means as text, 0.5 means half the width of the text
\newcommand{\figuremacroW}[4]{
	\begin{figure}[htbp]
		\centering
		\includegraphics[width=#4\textwidth]{#1}
		\caption[#2]{\textbf{#2} - #3}
		\label{#1}
	\end{figure}
}

% inserts a figure with wrapped around text; only suitable for NARROW figs
% o is for outside on a double paged document; others: l, r, i(inside)
% text and figure will each be half of the document width
% note: long captions often crash with adjacent content; take care
% in general: above 2 macro produce more reliable layout
\newcommand{\figuremacroN}[3]{
	\begin{wrapfigure}{o}{0.5\textwidth}
		\centering
		\includegraphics[width=0.48\textwidth]{#1}
		\caption[#2]{{\small\textbf{#2} - #3}}
		\label{#1}
	\end{wrapfigure}
}

% predefined commands by Harish
\newcommand{\PdfPsText}[2]{
  \ifpdf
     #1
  \else
     #2
  \fi
}

\newcommand{\IncludeGraphicsH}[3]{
  \PdfPsText{\includegraphics[height=#2]{#1}}{\includegraphics[bb = #3, height=#2]{#1}}
}

\newcommand{\IncludeGraphicsW}[3]{
  \PdfPsText{\includegraphics[width=#2]{#1}}{\includegraphics[bb = #3, width=#2]{#1}}
}

\newcommand{\InsertFig}[3]{
  \begin{figure}[!htbp]
    \begin{center}
      \leavevmode
      #1
      \caption{#2}
      \label{#3}
    \end{center}
  \end{figure}
}







%%% Local Variables:
%%% mode: latex
%%% TeX-master: "~/Documents/LaTeX/CUEDThesisPSnPDF/thesis"
%%% End:
           % Archivo con funciones útiles





%%%%%%%%%%%%%%%%%%%%%%%%%%%%%%%%%%%%%%%%%%%%%%%%%%%%%%%%%%%%%%%%%%%%%%%%%%%%%%%%
%                                   DATOS                                      %
%%%%%%%%%%%%%%%%%%%%%%%%%%%%%%%%%%%%%%%%%%%%%%%%%%%%%%%%%%%%%%%%%%%%%%%%%%%%%%%%
\title{Título de la tesis}
\author{Nombre Apellido1 Apellido 2} 
\facultad{Facultad de Algo Seguramente Muy Importante}                 % Nombre de la facultad/escuela
\escudofacultad{Latex/Classes/Escudos/fi_azul} % Aquí ponen la ruta y nombre del escudo de su facultad, actualmente, la carpeta Latex/Classes/Escudos cuenta con los siguientes escudos:
% "fi_azul" Facultad de ingenieria en color azul
% "fi_negro" Facultad de ingenieria en color negro
% "fc_azul" Facultad de ciencias en color azul
% "fc_negro" Facultad de ciencias en color negro
% "fmed_grande" Facultad de medicina UMSNH
% Se agradecen sus aportaciones de escudos a jebus.velazquez@gmail.com

\degree{Médic@/Ingenier@/Licenciad@}       % Carrera
\director{Nombre Director}                 % Director de tesis
%\tutor{Nombre  Tutor }                    % Tutor de tesis, si aplica
\degreedate{2020}                          % Año de la fecha del examen
\lugar{Ciudad Universitaria, CDMX}         % Lugar

%\portadafalse                              % Portada en NEGRO, descomentar y comentar la línea siguiente si se quiere utilizar
\portadatrue                                % Portada en COLOR



%% Opciones del posgrado (descomentar si las necesitan)
	%\posgradotrue                                                    
	%\programa{programa de maestría y doctorado en ingeniería}
	%\campo{Ingeniería Eléctrica - Control}
	%% En caso de que haya comité tutor
	%\comitetrue
	%\ctutoruno{Dr. Emmet L. Brown}
	%\ctutordos{Dr. El Doctor}
%% Datos del jurado                             
	%\presidente{Dr. 1}
	%\secretario{Dr. 2}
	%\vocal{Dr. 3}
	%\supuno{Dr. 4}
	%\supdos{Dr. 5}
	%\institucion{el Instituto de Ingeniería, UNAM}

\keywords{tesis,autor,tutor,etc}            % Palablas clave para los metadatos del PDF
\subject{tema_1,tema_2}                     % Tema para metadatos del PDF  

%%%%%%%%%%%%%%%%%%%%%%%%%%%%%%%%%%%%%%%%%%%%%%%%%%%%%
%                   PORTADA                         %
%%%%%%%%%%%%%%%%%%%%%%%%%%%%%%%%%%%%%%%%%%%%%%%%%%%%%
\begin{document}

\maketitle									% Se redefinió este comando en el archivo de la clase para generar automáticamente la portada a partir de los datos

%%%%%%%%%%%%%%%%%%%%%%%%%%%%%%%%%%%%%%%%%%%%%%%%%%%%%
%                  PRÓLOGO                          %
%%%%%%%%%%%%%%%%%%%%%%%%%%%%%%%%%%%%%%%%%%%%%%%%%%%%%
\frontmatter
\include{Agradecimientos/Dedicatoria}       % Comentar línea si no se usa
\include{Agradecimientos/Agradecimientos}   % Comentar línea si no se usa 
\include{Declaracion/Declaracion}           % Comentar línea si no se usa
\include{Resumen/Resumen}                   % Comentar línea si no se usa

%%%%%%%%%%%%%%%%%%%%%%%%%%%%%%%%%%%%%%%%%%%%%%%%%%%%%
%                   ÍNDICES                         %
%%%%%%%%%%%%%%%%%%%%%%%%%%%%%%%%%%%%%%%%%%%%%%%%%%%%%
%Esta sección genera el índice
\setcounter{secnumdepth}{3} % organisational level that receives a numbers
\setcounter{tocdepth}{3}    % print table of contents for level 3
\tableofcontents            % Genera el índice 
%: ----------------------- list of figures/tables ------------------------
\listoffigures              % Genera el ínidce de figuras, comentar línea si no se usa
\listoftables               % Genera índice de tablas, comentar línea si no se usa


%%%%%%%%%%%%%%%%%%%%%%%%%%%%%%%%%%%%%%%%%%%%%%%%%%%%%
%                   CONTENIDO                       %
%%%%%%%%%%%%%%%%%%%%%%%%%%%%%%%%%%%%%%%%%%%%%%%%%%%%%
% the main text starts here with the introduction, 1st chapter,...
\mainmatter
\def\baselinestretch{1.5}                   % Interlineado de 1.5
\include{Capitulo1/introduccion}            % ~10 páginas - Explicar el propósito de la tesis
\include{Capitulo2/marco_teorico}           % ~20 páginas - Poner un contexto a la tesis, hacer referencia a trabajos actuales en el tema
\include{Capitulo3/diseno_experimento}      % ~20 páginas - Explicar el problema en específico que se va a resolver, la metodología y experimentos/métodos utilizados
\include{Capitulo4/resultados_y_analisis}   % ~20 páginas - Presentar los resultados tal cual son, y analizarlos.
\include{Capitulo5/conclusiones}            % ~5 páginas - Resumir lo que se hizo y lo que no y comentar trabajos futuros sobre el tema

%%%%%%%%%%%%%%%%%%%%%%%%%%%%%%%%%%%%%%%%%%%%%%%%%%%%%
%                   APÉNDICES                       %
%%%%%%%%%%%%%%%%%%%%%%%%%%%%%%%%%%%%%%%%%%%%%%%%%%%%%
\appendix
\include{Apendice1/Apendice1}               % Colocar los circuitos, manuales, código fuente, pruebas de teoremas, etc.

%%%%%%%%%%%%%%%%%%%%%%%%%%%%%%%%%%%%%%%%%%%%%%%%%%%%%
%                   REFERENCIAS                     %
%%%%%%%%%%%%%%%%%%%%%%%%%%%%%%%%%%%%%%%%%%%%%%%%%%%%%
% existen varios estilos de bilbiografía, pueden cambiarlos a placer
\bibliographystyle{apalike} % otros estilos pueden ser abbrv, acm, alpha, apalike, ieeetr, plain, siam, unsrt

%El formato trae otros estilos, o pueden agregar uno que les guste:
%\bibliographystyle{Latex/Classes/PhDbiblio-case} % title forced lower case
%\bibliographystyle{Latex/Classes/PhDbiblio-bold} % title as in bibtex but bold
%\bibliographystyle{Latex/Classes/PhDbiblio-url} % bold + www link if provided
%\bibliographystyle{Latex/Classes/jmb} % calls style file jmb.bst

\bibliography{Bibliografia/referencias}             % Archivo .bib


\end{document}
