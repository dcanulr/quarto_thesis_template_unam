% Options for packages loaded elsewhere
\PassOptionsToPackage{unicode}{hyperref}
\PassOptionsToPackage{hyphens}{url}
%
\documentclass[
  letterpaper,
  letterpaper,
  twocolumn]{./tex/tesisIER}

\usepackage{amsmath,amssymb}
\usepackage{iftex}
\ifPDFTeX
  \usepackage[T1]{fontenc}
  \usepackage[utf8]{inputenc}
  \usepackage{textcomp} % provide euro and other symbols
\else % if luatex or xetex
  \usepackage{unicode-math}
  \defaultfontfeatures{Scale=MatchLowercase}
  \defaultfontfeatures[\rmfamily]{Ligatures=TeX,Scale=1}
\fi
\usepackage{lmodern}
\ifPDFTeX\else  
    % xetex/luatex font selection
\fi
% Use upquote if available, for straight quotes in verbatim environments
\IfFileExists{upquote.sty}{\usepackage{upquote}}{}
\IfFileExists{microtype.sty}{% use microtype if available
  \usepackage[]{microtype}
  \UseMicrotypeSet[protrusion]{basicmath} % disable protrusion for tt fonts
}{}
\makeatletter
\@ifundefined{KOMAClassName}{% if non-KOMA class
  \IfFileExists{parskip.sty}{%
    \usepackage{parskip}
  }{% else
    \setlength{\parindent}{0pt}
    \setlength{\parskip}{6pt plus 2pt minus 1pt}}
}{% if KOMA class
  \KOMAoptions{parskip=half}}
\makeatother
\usepackage{xcolor}
\setlength{\emergencystretch}{3em} % prevent overfull lines
\setcounter{secnumdepth}{5}
% Make \paragraph and \subparagraph free-standing
\ifx\paragraph\undefined\else
  \let\oldparagraph\paragraph
  \renewcommand{\paragraph}[1]{\oldparagraph{#1}\mbox{}}
\fi
\ifx\subparagraph\undefined\else
  \let\oldsubparagraph\subparagraph
  \renewcommand{\subparagraph}[1]{\oldsubparagraph{#1}\mbox{}}
\fi


\providecommand{\tightlist}{%
  \setlength{\itemsep}{0pt}\setlength{\parskip}{0pt}}\usepackage{longtable,booktabs,array}
\usepackage{calc} % for calculating minipage widths
% Correct order of tables after \paragraph or \subparagraph
\usepackage{etoolbox}
\makeatletter
\patchcmd\longtable{\par}{\if@noskipsec\mbox{}\fi\par}{}{}
\makeatother
% Allow footnotes in longtable head/foot
\IfFileExists{footnotehyper.sty}{\usepackage{footnotehyper}}{\usepackage{footnote}}
\makesavenoteenv{longtable}
\usepackage{graphicx}
\makeatletter
\def\maxwidth{\ifdim\Gin@nat@width>\linewidth\linewidth\else\Gin@nat@width\fi}
\def\maxheight{\ifdim\Gin@nat@height>\textheight\textheight\else\Gin@nat@height\fi}
\makeatother
% Scale images if necessary, so that they will not overflow the page
% margins by default, and it is still possible to overwrite the defaults
% using explicit options in \includegraphics[width, height, ...]{}
\setkeys{Gin}{width=\maxwidth,height=\maxheight,keepaspectratio}
% Set default figure placement to htbp
\makeatletter
\def\fps@figure{htbp}
\makeatother

\makeatletter
\makeatother
\makeatletter
\@ifpackageloaded{bookmark}{}{\usepackage{bookmark}}
\makeatother
\makeatletter
\@ifpackageloaded{caption}{}{\usepackage{caption}}
\AtBeginDocument{%
\ifdefined\contentsname
  \renewcommand*\contentsname{Tabla de contenidos}
\else
  \newcommand\contentsname{Tabla de contenidos}
\fi
\ifdefined\listfigurename
  \renewcommand*\listfigurename{Listado de Figuras}
\else
  \newcommand\listfigurename{Listado de Figuras}
\fi
\ifdefined\listtablename
  \renewcommand*\listtablename{Listado de Tablas}
\else
  \newcommand\listtablename{Listado de Tablas}
\fi
\ifdefined\figurename
  \renewcommand*\figurename{Figura}
\else
  \newcommand\figurename{Figura}
\fi
\ifdefined\tablename
  \renewcommand*\tablename{Tabla}
\else
  \newcommand\tablename{Tabla}
\fi
}
\@ifpackageloaded{float}{}{\usepackage{float}}
\floatstyle{ruled}
\@ifundefined{c@chapter}{\newfloat{codelisting}{h}{lop}}{\newfloat{codelisting}{h}{lop}[chapter]}
\floatname{codelisting}{Listado}
\newcommand*\listoflistings{\listof{codelisting}{Listado de Listados}}
\makeatother
\makeatletter
\@ifpackageloaded{caption}{}{\usepackage{caption}}
\@ifpackageloaded{subcaption}{}{\usepackage{subcaption}}
\makeatother
\makeatletter
\@ifpackageloaded{tcolorbox}{}{\usepackage[skins,breakable]{tcolorbox}}
\makeatother
\makeatletter
\@ifundefined{shadecolor}{\definecolor{shadecolor}{rgb}{.97, .97, .97}}
\makeatother
\makeatletter
\makeatother
\makeatletter
\makeatother
\ifLuaTeX
\usepackage[bidi=basic]{babel}
\else
\usepackage[bidi=default]{babel}
\fi
\babelprovide[main,import]{spanish}
% get rid of language-specific shorthands (see #6817):
\let\LanguageShortHands\languageshorthands
\def\languageshorthands#1{}
\ifLuaTeX
  \usepackage{selnolig}  % disable illegal ligatures
\fi
\IfFileExists{bookmark.sty}{\usepackage{bookmark}}{\usepackage{hyperref}}
\IfFileExists{xurl.sty}{\usepackage{xurl}}{} % add URL line breaks if available
\urlstyle{same} % disable monospaced font for URLs
\hypersetup{
  pdftitle={Analisis del clima en Temixco},
  pdfauthor={Guillermo Barrios del Valle},
  pdflang={es},
  hidelinks,
  pdfcreator={LaTeX via pandoc}}

\title{Analisis del clima en Temixco}
\author{Guillermo Barrios del Valle}
\date{2024-02-20}

\begin{document}
\maketitle
\ifdefined\Shaded\renewenvironment{Shaded}{\begin{tcolorbox}[sharp corners, frame hidden, interior hidden, breakable, enhanced, borderline west={3pt}{0pt}{shadecolor}, boxrule=0pt]}{\end{tcolorbox}}\fi

\renewcommand*\contentsname{Tabla de contenidos}
{
\setcounter{tocdepth}{2}
\tableofcontents
}
\bookmarksetup{startatroot}

\hypertarget{abstract}{%
\chapter*{Abstract}\label{abstract}}
\addcontentsline{toc}{chapter}{Abstract}

\markboth{Abstract}{Abstract}

Este es el documento index.qmd

\bookmarksetup{startatroot}

\hypertarget{prefacio}{%
\chapter{Prefacio}\label{prefacio}}

\bookmarksetup{startatroot}

\hypertarget{windrose}{%
\chapter{windrose}\label{windrose}}

Quarto o Quarto! es un juego de mesa, dentro de la categoría de los
juegos abstractos de estrategia por turnos, creado por el matemático
suizo Blaise Muller, premiado en 1985 en el «Concurso internacional» de
creadores de juegos de mesa con el nombre de 4x41\hspace{0pt} y editado
desde 1991 por Gigamic.

Se trata de un juego de estrategia por turnos, para dos personas. Es
conceptualmente muy sencillo, pero tiene una enorme cantidad de
posibilidades.

Presentación El objetivo del juego es lograr alinear 4 piezas, pero uno
no juega las piezas que elige, sino el adversario es quien las escoge
por uno.

Las dieciséis piezas del juego, todas diferentes, tienen cada una 4
características distintas: alta o baja, redonda o cuadrada, clara u
obscura, plana o tallada. Cada jugador, a su vez, escoge una pieza y la
da a su adversario, que debe colocarla en una casilla vacía. El ganador
es aquel que, con una pieza recibida, crea una alineación de 4 piezas
con al menos una característica común y anuncia: «quarto!».

Composición El juego se compone de:

Un tablero de 16 casillas (4*4) 16 peones diferenciables por 4
características: El color - claro / oscuro La altura - pequeño / grande
La punta - plena / agujereada La forma - paralelepípeda / cilíndrica
Todas las combinaciones (ejemplo: grande, clara, agujereada y
cilíndrica) están representadas, y una sola vez cada una. Un estuche
para guardar las piezas Turno de juego Cada turno consiste en dar al
adversario un peón para que lo coloque donde quiera sobre el tablero.

El objetivo del juego La regla básica es conformar una línea de cuatro
peones con una característica común y anunciarlo. Sin embargo, si se
juega sistemáticamente de manera defensiva, las partidas terminan por lo
general en posición de empate, por lo que las reglas prevén otras
configuraciones ganadoras, siguiendo cuatro niveles de juego.

La regla básica corresponde al nivel uno, la victoria se obtiene
realizando un alineamiento. El nivel dos permite ganar con un
alineamiento o un pequeño cuadrado. El nivel tres, además de las
posiciones ganadoras del nivel 2, también se obtiene la victoria con un
cuadrado más grande. El cuarto nivel, agrega a las condiciones de los
niveles anteriores, los cuadrados «móviles» (de mayor dificultad). Al
nivel cuatro de juego, se eliminan prácticamente todas las posiciones de
igualdad. Además, la existencia de diferentes niveles permite introducir
una variante adicional: los dos jugadores no juegan al mismo nivel y por
lo tanto no tienen los mismos esquemas ganadores.

Forma de juego alternativa Una variante del juego es que al formar una
línea de cuatro peones con una característica común el jugador pierda,
por lo tanto se debe evitar y gana el jugador que obliga a su oponente a
formar una línea de cuatro peones con una característica común como
única jugada posible. Este modo de juego es más complicado porque se
debe pensar en el futuro, teniendo en cuenta las piezas que restan sin
usar.

\bookmarksetup{startatroot}

\hypertarget{referencias}{%
\chapter{referencias}\label{referencias}}



\end{document}
